\documentclass[uplatex, dvipdfmx, fleqn, a4paper, 10pt]{ujreport}
\usepackage{doc}

\everymath{\displaystyle}

\begin{document}

\chapter{アルゴリズムとプログラミング}
アルゴリズム設計,手続き型プログラム,計算量,データ構造,再帰,整列アルゴリズム,探索アルゴリズム
\newpage
\chapter{計算機システムとシステムプログラム}
計算機システム分野:
数の表現,演算制御,命令実行制御,記憶制御,入出力制御
システムプログラム分野:
プロセス管理,処理装置管理,記憶管理,入出力管理,ファイル管理
\newpage
\chapter{離散構造}
集合・命題,関係,漸化式,論理関数,ブール代数,最簡積和形,命題論理,述語論理,導出原理,グラフ
\newpage
\chapter{計算理論}
語・言語,有限オートマトン,正規表現・言語,形式文法とそのクラス,導出・認識・構文解析,文脈自由文法・言語,プッシュダウンオートマトン
\newpage
\chapter{ネットワーク}
情報源符号化・通信路符号化,階層化モデル,プロトコルとインターフェース,各層プロトコルの設計・仕様・評価手法,ネットワークアプリケーション
\newpage
\chapter{電子回路と論理設計}
ダイオード・トランジスタ,MOSFET,アナログ電子回路,演算増幅器,記憶素子,数の表現,論理代数と論理関数,組合せ論理回路,順序回路,算術演算回路
\newpage
\chapter{数学解析と信号処理}

\begin{minipage}{0.33\linewidth}
    \begin{itemize}
        \item 微分方程式
        \item フーリエ級数
        \item ラプラス変換
        \item Z変換
    \end{itemize}
\end{minipage}
\hspace{0.05\linewidth}
\begin{minipage}{0.33\linewidth}
    \begin{itemize}
        \item 連続時間信号のフーリエ解析
        \item 離散時間信号のフーリエ解析
        \item 複素関数
    \end{itemize}
\end{minipage}
\hspace{0.05\linewidth}
\begin{minipage}{0.33\linewidth}
    \begin{itemize}
        \item 信号の演算
        \item サンプリング
        \item フィルタ
    \end{itemize}
\end{minipage}

\section{ラプラス変換}\label{sec:laplace_transform}

\begin{defbox}{ラプラス変換}
    $t \ge 0 < \infty$の連続関数$f(t)$について,
    \begin{eqnarray}
        F(s) = \mathcal{L}[f(t)](s) = \int_{0}^{\infty} f(t)e^{-st} dt
        \label{eq:laplace_transform}
    \end{eqnarray}
    が収束するとき,$F(s)$を$f(t)$の\textbf{ラプラス変換}という.
\end{defbox}

\subsection{代表的なラプラス変換}

\begin{exprbox}{指数関数のラプラス変換}
    $s > a$のとき,
    \begin{eqnarray}
        \mathcal{L}[e^{at}](s) = \frac{1}{s - a}
    \end{eqnarray}
    \begin{proof}
        \begin{eqnarray*}
            \mathcal{L}[e^{at}](s) &=& \int_{0}^{\infty} e^{at} e^{-st} dt
            = \lim_{T \to \infty} \int_{0}^{T} e^{-(s - a)t} dt \\
            &=& \lim_{T \to \infty} \left[-\frac{1}{s - a}e^{-(s - a)t}\right]_0^T 
            = \lim_{T \to \infty} -\frac{1}{s - a}e^{-(s - a)T} - \left(-\frac{1}{s - a}\right) \\
            &=& \frac{1}{s - a}
        \end{eqnarray*}
    \end{proof}
\end{exprbox}

\begin{exprbox}{三角関数のラプラス変換}
    $a$を実定数として,
    \begin{eqnarray}
        \mathcal{L}[\cos at](s) = \frac{s}{s^2 + a^2},\hspace{10pt}\mathcal{L}[\sin at](s) = \frac{a}{s^2 + a^2}
    \end{eqnarray}
    \begin{proof}
        $e^{ia} = \cos a + i\sin a$より,
        \begin{eqnarray}
            \cos a = \frac{e^{ia} + e^{-ia}}{2},\hspace{10pt}\sin a = \frac{e^{ia} - e^{-ia}}{2i}
        \end{eqnarray}
        が成り立つ.
        \begin{eqnarray*}
            \mathcal{L}[\cos at](s) &=& \mathcal{L}[\frac{e^{iat} + e^{-iat}}{2}](s)
            = \frac{1}{2}\left(\mathcal{L}[e^{iat}](s) + \mathcal{L}[e^{-iat}](s)\right)\\
            &=& \frac{1}{2} \left(\frac{1}{s - ia} + \frac{1}{s + ia}\right)
            = \frac{1}{2} \frac{2s}{s^2 - (ia)^2} \\
            &=& \frac{s}{s^2 + a^2}
        \end{eqnarray*}
        同様に
        \begin{eqnarray*}
            \mathcal{L}[\sin at](s) &=& \mathcal{L}[\frac{e^{iat} - e^{-iat}}{2i}](s)
            = \frac{1}{2i}\left(\mathcal{L}[e^{iat}](s) - \mathcal{L}[e^{-iat}](s)\right)\\
            &=& \frac{1}{2i} \left(\frac{1}{s - ia} - \frac{1}{s + ia}\right)
            = \frac{1}{2i} \frac{2ia}{s^2 - (ia)^2} \\
            &=& \frac{a}{s^2 + a^2}
        \end{eqnarray*}
    \end{proof}
\end{exprbox}

\begin{exprbox}{双曲線関数のラプラス変換}
    $a$を実定数として,
    \begin{eqnarray}
        \mathcal{L}[\cosh at](s) = \frac{s}{s^2 - a^2},\hspace{10pt}\mathcal{L}[\sinh at](s) = \frac{a}{s^2 - a^2}
    \end{eqnarray}
    \begin{proof}
        双曲線関数を以下のように定義する.
        \begin{eqnarray}
            \cosh a = \frac{e^{a} + e^{-a}}{2},\hspace{10pt}\sinh a = \frac{e^{a} - e^{-a}}{2}
        \end{eqnarray}
        \begin{eqnarray*}
            \mathcal{L}[\cosh at](s) &=& \mathcal{L}[\frac{e^{at} + e^{-at}}{2}](s)
            = \frac{1}{2}\left(\mathcal{L}[e^{at}](s) + \mathcal{L}[e^{-at}](s)\right)\\
            &=& \frac{1}{2} \left(\frac{1}{s - a} + \frac{1}{s + a}\right)
            = \frac{1}{2} \frac{2s}{s^2 - a^2} \\
            &=& \frac{s}{s^2 - a^2}
        \end{eqnarray*}
        同様に
        \begin{eqnarray*}
            \mathcal{L}[\sinh at](s) &=& \mathcal{L}[\frac{e^{at} - e^{-at}}{2}](s)
            = \frac{1}{2}\left(\mathcal{L}[e^{at}](s) - \mathcal{L}[e^{-at}](s)\right)\\
            &=& \frac{1}{2} \left(\frac{1}{s - a} - \frac{1}{s + a}\right)
            = \frac{1}{2} \frac{2a}{s^2 - a^2} \\
            &=& \frac{a}{s^2 - a^2}
        \end{eqnarray*}
    \end{proof}
\end{exprbox}

\begin{exprbox}{多項式のラプラス変換}
    \begin{eqnarray}
        \mathcal{L}[t^n](s) = \frac{n!}{s^{n + 1}}
    \end{eqnarray}
    \begin{proof}
        $F_n(s) = \mathcal{L}[t^n](s)$と定義して,
        \begin{eqnarray*}
            F_n(s) &=& \int_{0}^{\infty} t^n e^{-st} dt
            = \left[t^n \left(-\frac{1}{s} e^{-st}\right)\right]_0^\infty - \int_{0}^{\infty} n t^{n - 1} \left(-\frac{1}{s} e^{-st}\right) dt \\
            &=& \frac{n}{s} \int_{0}^{\infty} t^{n - 1} e^{-st} dt = \frac{n}{s} \mathcal{L}[t^{n - 1}](s)
            = \frac{n}{s} F_{n - 1}(s) \\
            &=& \frac{n}{s} \frac{n - 1}{s} \cdots \frac{1}{s} F_0(s) \\
            F_0(s) &=& \mathcal{L}[t^0](s) = \int_{0}^{\infty} e^{-st} dt \\
            &=& \left[-\frac{1}{s}e^{-st}\right]_0^\infty = \frac{1}{s} \\
            \therefore F_n(s) &=& \frac{n!}{s^{n + 1}}
        \end{eqnarray*}
    \end{proof}
\end{exprbox}

\newpage

\subsection{ラプラス変換の性質}

\begin{exprbox}{線形性}
    連続関数f(t),g(t)がラプラス変換可能なら,次が成り立つ.
    \begin{eqnarray}
        \mathcal{L}[a f(t) + b g(t)](s) = a\mathcal{L}[f(t)](s) + b\mathcal{L}[g(t)](s)
    \end{eqnarray}
    \begin{proof}
        \begin{eqnarray*}
            \mathcal{L}[a f(t) + b g(t)](s) &=& 
            \int_{0}^{\infty} \left(a f(t) + b g(t)\right) e^{-st} dt\\
            &=& a \int_{0}^{\infty} f(t) e^{-st} dt + b \int_{0}^{\infty} g(t) e^{-st} dt \\
            &=& a\mathcal{L}[f(t)](s) + b\mathcal{L}[g(t)](s)
        \end{eqnarray*}
    \end{proof}
\end{exprbox}

\begin{exprbox}{像の移動法則}
    \begin{eqnarray}
        \mathcal{L}[e^{at}f(t)](s) = F(s - a)
    \end{eqnarray}
    \begin{proof}
        \begin{eqnarray*}
            \mathcal{L}[e^{at}f(t)](s) &=& \int_{0}^{\infty} e^{at} f(t) e^{-st} dt
            = \int_{0}^{\infty} f(t) e^{-(s - a)t} dt \\
            &=& \mathcal{L}[f(t)](s - a) = F(s - a)
        \end{eqnarray*}
    \end{proof}
\end{exprbox}

\begin{exprbox}{微分法則}
    \begin{eqnarray}
        \mathcal{L}[f'(t)](s) &=& sF(s) - f(0) \\
        \mathcal{L}[f''(t)](s) &=& s^2F(s) - sf(0) - f'(0) \\
        \vdots \\
        \mathcal{L}[f^{(n)}(t)](s) &=& s^nF(s) - s^{n - 1}f(0) - s^{n - 2}f'(0) - \cdots - f^{(n)}(s)
    \end{eqnarray}
    \begin{proof}
        \begin{eqnarray*}
            \mathcal{L}[f'(t)](s) &=& \int_{0}^{\infty} f'(t) e^{-st} dt
            = \left[f(t) e^{-st}\right]_0^\infty - \int_{0}^{\infty} f(t) se^{-st} dt
            = -f(0) + s \int_{0}^{\infty} f(t) e^{-st} dt\\
            &=& sF(s) - f(0)
        \end{eqnarray*}
        \begin{eqnarray*}
            \mathcal{L}[f''(t)](s) &=& \mathcal{L}[\left(f'(t)\right)'](s) \\
            &=& s \mathcal{L}[f'(t)](s) - f'(0) = s\left(sF(s) - f(0)\right) - f'(0) \\
            &=& s^2 F(s) - s f(0) - f'(0)
        \end{eqnarray*}
        同様に
        \begin{eqnarray*}
            \mathcal{L}[f^{(n)}(t)](s) &=& s^nF(s) - s^{n - 1}f(0) - s^{n - 2}f'(0) - \cdots - f^{(n)}(s)
        \end{eqnarray*}
    \end{proof}
\end{exprbox}

\begin{exprbox}{像の微分法則}
    \begin{eqnarray}
        \mathcal{L}[t f(t)](s) = -\frac{d}{ds} F(s),\hspace{10pt}\mathcal{L}[t^n f(t)] = \left(-\frac{d}{ds}\right)^n F(s)
        \label{eq:laplace_image_derivative}
    \end{eqnarray}
    \begin{proof}
        $F_n(s) = \left(-\frac{d}{ds}\right)^n F_0(s)$として,
        \begin{eqnarray*}
            F_1(s) &=& -\frac{d}{ds}F_0(s) = \int_{0}^{\infty} -\frac{\partial}{\partial s} \left\{e^{-st} f(t)\right\} dt
            = \int_{0}^{\infty} t f(t) e^{-st} dt = \mathcal{L}[tf(t)](s) \\
            F_2(s) &=& \left(-\frac{d}{ds}\right) F_1(s) 
            = \int_{0}^{\infty} -\frac{\partial}{\partial s} \left\{t f(t) e^{-st}\right\} dt
            = \int_{0}^{\infty} t^2 f(t) e^{-st} dt = \mathcal{L}[t^2f(t)](s)
        \end{eqnarray*}
        同様に
        \begin{eqnarray*}
            F_n(s) = \mathcal{L}[t^nf(t)](s)
        \end{eqnarray*}
    \end{proof}
\end{exprbox}

\begin{exprbox}{相似法則}
    \begin{eqnarray}
        \mathcal{L}[f(at)](s) = \frac{1}{a}F\left(\frac{s}{a}\right)
    \end{eqnarray}
    \begin{proof}
        \begin{eqnarray*}
            \mathcal{L}[f(at)](s) = \int_{0}^{\infty} f(at) e^{-st} dt
            \overset{u = at}{=} \int_{0}^{\infty} f(u) e^{-\frac{s}{a}u} \frac{dt}{a}
            = \frac{1}{a} F\left(\frac{s}{a}\right)
        \end{eqnarray*}
    \end{proof}
\end{exprbox}

\begin{exprbox}{たたみ込み}
    畳み込みを次の式で定義する.
    \begin{eqnarray}
        f \ast g(t) = \int_{0}^{t} f(\tau)g(t - \tau) d\tau
    \end{eqnarray}
    このとき,畳み込みのラプラス変換は次のようにラプラス変換の積となる.
    \begin{eqnarray}
        \mathcal{L}[f \ast g(t)](s) = F(s)G(s)
    \end{eqnarray}
    \begin{proof}
        \begin{eqnarray*}
            F(s)G(s) &=& G(s) \int_{0}^{\infty} f(\tau) e^{-s\tau} d\tau
            = \int_{0}^{\infty} f(\tau) G(s) e^{-s\tau} d\tau
        \end{eqnarray*}
        ここで,
        \begin{eqnarray*}
            G(s) e^{-s\tau} &=& \int_{0}^{\infty} e^{-sx} g(x) dx e^{-s\tau}
            = \int_{0}^{\infty} g(x) e^{-s(x+\tau)} dx \\
            &\overset{t = x + \tau}{=}& \int_{\tau}^{\infty} g(t - \tau) e^{-st} dt
        \end{eqnarray*}
        なので,もとの式に代入して,
        \begin{eqnarray*}
            F(s)G(s) &=& \int_{0}^{\infty} f(\tau) \left\{\int_{\tau}^{\infty} g(t - \tau) e^{-st} dt\right\} d\tau
        \end{eqnarray*}
        領域$D: t \ge \tau, \tau \ge 0$は領域$D': 0 \le \tau \le t, t \ge 0$と等しいため,次のようになる.
        \begin{eqnarray*}
            F(s)G(s) &=& \int_{0}^{\infty} \int_{0}^{t} f(\tau) g(t - \tau) e^{-st} d\tau dt\\
            &=& \int_{0}^{\infty} \left\{\int_{0}^{t} f(\tau) g(t - \tau) d\tau\right\} dt
            = \mathcal{L}[f \ast g(t)](s)
        \end{eqnarray*}
    \end{proof}
\end{exprbox}

\subsection{ラプラス変換表}

\begin{minipage}{0.5\linewidth}
    \centering
    \begin{tabular*}{6.88cm}{wc{3cm}wc{3cm}} \toprule
        $1$ & $\frac{1}{s}$ \\ 
        $t^n$ & $\frac{n!}{s^{n+1}}$ \\
        $t^\alpha$ & $\frac{\Gamma(\alpha + 1)}{s^{\alpha+1}}$ \\\bottomrule
    \end{tabular*}
\end{minipage}
\begin{minipage}{0.5\linewidth}
    \begin{tabular*}{6.88cm}{wc{3cm}wc{3cm}} \toprule
        $1$ & $\frac{1}{s}$ \\ \bottomrule
    \end{tabular*}
\end{minipage}

\subsection{ラプラス変換が存在する条件}

\subsection{ラプラス変換の応用}

\subsubsection{特殊な積分}

ラプラス変換を用いることで,いくつかの特殊な積分を簡単に解ける.
以下に例を示す.

\begin{exprbox}{ディリクレ積分}
    \begin{eqnarray}
        \int_{0}^{\infty} \frac{\sin t}{t} dt = \frac{\pi}{2}
    \end{eqnarray}
    \begin{proof}
        \begin{eqnarray*}
            J(s) = \mathcal{L}\left[\frac{\sin t}{t}\right] = \int_{0}^{\infty} \frac{\sin t}{t} e^{-st} dt
        \end{eqnarray*}
        と定義する.像の微分法則(\eref{eq:laplace_image_derivative})より,
        \begin{eqnarray*}
            \frac{dJ}{ds}(s) = - \mathcal{L}\left[t \frac{\sin t}{t}\right](s) = -\mathcal{L}[\sin t](s) = - \frac{1}{s^2 + 1}
        \end{eqnarray*}
        となるので,任意定数$C$について次の式が成り立つ.
        \begin{eqnarray*}
            J(s) = -\arcsin s + C
        \end{eqnarray*}
        $\lim_{s \to \infty} J(s) = 0$,$\lim_{s \to \infty} = \frac{\pi}{2}$より,$C = \frac{\pi}{2}$である.
        したがって,$J(0) = \frac{\pi}{2}$より,
        \begin{eqnarray*}
            \int_{0}^{\infty} \frac{\sin t}{t} dt = \frac{\pi}{2}
        \end{eqnarray*}
    \end{proof}
\end{exprbox}

\subsubsection{常微分方程式}

\subsection{まとめ}

\section{z変換}\label{sec:z_transform}

\section{フーリエ変換}\label{sec:fourier_transform}

\section{変換表}

\newpage
\end{document}