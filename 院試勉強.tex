\documentclass[uplatex, dvipdfmx, fleqn, a4paper, 10pt]{ujreport}
\usepackage{doc}

\begin{document}

\chapter{アルゴリズムとプログラミング}
アルゴリズム設計,手続き型プログラム,計算量,データ構造,再帰,整列アルゴリズム,探索アルゴリズム
\newpage
\chapter{計算機システムとシステムプログラム}
計算機システム分野:
数の表現,演算制御,命令実行制御,記憶制御,入出力制御
システムプログラム分野:
プロセス管理,処理装置管理,記憶管理,入出力管理,ファイル管理
\newpage
\chapter{離散構造}
集合・命題,関係,漸化式,論理関数,ブール代数,最簡積和形,命題論理,述語論理,導出原理,グラフ
\newpage
\chapter{計算理論}
語・言語,有限オートマトン,正規表現・言語,形式文法とそのクラス,導出・認識・構文解析,文脈自由文法・言語,プッシュダウンオートマトン
\newpage
\chapter{ネットワーク}
情報源符号化・通信路符号化,階層化モデル,プロトコルとインターフェース,各層プロトコルの設計・仕様・評価手法,ネットワークアプリケーション
\newpage
\chapter{電子回路と論理設計}
ダイオード・トランジスタ,MOSFET,アナログ電子回路,演算増幅器,記憶素子,数の表現,論理代数と論理関数,組合せ論理回路,順序回路,算術演算回路
\newpage
\chapter{数学解析と信号処理}

\begin{minipage}{0.33\linewidth}
    \begin{itemize}
        \item 微分方程式
        \item フーリエ級数
        \item ラプラス変換
        \item Z変換
    \end{itemize}
\end{minipage}
\hspace{0.05\linewidth}
\begin{minipage}{0.33\linewidth}
    \begin{itemize}
        \item 連続時間信号のフーリエ解析
        \item 離散時間信号のフーリエ解析
        \item 複素関数
    \end{itemize}
\end{minipage}
\hspace{0.05\linewidth}
\begin{minipage}{0.33\linewidth}
    \begin{itemize}
        \item 信号の演算
        \item サンプリング
        \item フィルタ
    \end{itemize}
\end{minipage}

\section{ラプラス変換}\label{sec:laplace_transform}

\begin{defbox}{ラプラス変換}
    $t \ge 0 < \infty$の連続関数$f(t)$について,
    \begin{eqnarray}
        F(s) = \mathcal{L}[f(t)](s) = \int_{0}^{\infty} f(t)e^{-st} dt
        \label{eq:laplace_transform}
    \end{eqnarray}
    が収束するとき,$F(s)$を$f(t)$の\textbf{ラプラス変換}という.
\end{defbox}

\subsection{代表的なラプラス変換}

\begin{exprbox}{指数関数のラプラス変換}
    \begin{eqnarray}
        \mathcal{L}[e^{at}](s) = \frac{1}{s - a}
    \end{eqnarray}
    \begin{proof}
        \begin{eqnarray*}
            \mathcal{L}[e^{at}](s) &=& \int_{0}^{\infty} e^{at} e^{-st} dt \\
            &=& \lim_{T \to \infty} \int_{0}^{T} e^{(a - s)t} dt \\
            &=& \left[\frac{1}{a - s}e^{(a - s)t}\right]_0^T \\
            &=& 
        \end{eqnarray*}
    \end{proof}
\end{exprbox}

\subsection{ラプラス変換の性質}

\subsection{ラプラス変換表}

\subsection{ラプラス変換が存在する条件}

\subsection{まとめ}

\section{z変換}\label{sec:z_transform}

\section{フーリエ変換}\label{sec:fourier_transform}

\section{変換表}

\newpage
\end{document}